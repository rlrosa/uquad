\documentclass[spanish,12pt,a4paper,titlepage]{report}
\usepackage[utf8]{inputenc}
\usepackage{graphicx}
\usepackage{subfig}
\usepackage{float}
\usepackage{wrapfig}
\usepackage{multirow}
\usepackage{caption}
\usepackage[spanish]{babel}
\usepackage[dvips]{hyperref}
\usepackage{amssymb}
\usepackage{listings}
\usepackage{epsfig}
\usepackage{amsmath}
\usepackage{array}
\usepackage[table]{xcolor}
\usepackage{multirow}
%\usepackage[Sonny]{fncychap}
\usepackage[Lenny]{fncychap}
%\usepackage[Glenn]{fncychap}
%\usepackage[Conny]{fncychap}
%\usepackage[Rejne]{fncychap}
%\usepackage[Bjarne]{fncychap}
%\usepackage[Bjornstrup]{fncychap}

%\usepackage{subfiles}
%\usepackage{framed}

\setlength{\topmargin}{-1.5cm}
\setlength{\textheight}{25cm}
\setlength{\oddsidemargin}{0.3cm} 
\setlength{\textwidth}{15cm}
\setlength{\columnsep}{0cm}

\begin{document}

\chapter{Calibraci�n de aceler�metro}

\section{Objetivos}

Se realiza una serie de pruebas con el fin de calibrar el aceler�metro de tres ejes de la IMU. Se estudiar�n diversos modelos posibles para dicha calibraci�n (lineal, c�bico,etc). Se estimaran adem�s las fuentes de error del mismo (ruido y bias).

\section{Materiales}
\begin{itemize}
\item IMU
\item Beagleboard
\item Tocadiscos
\item Cron�metro
\item Regla
\item Resorte
\item balanza
\end{itemize}
\section{Procedimiento}

\subsection{Parte I: Calibraci�n}
Se realizan dos experimentos a fin de obtener una serie de datos suficientemente diversa como para obtener una calibraci�n aceptable.

\subsubsection*{Medidas en reposo}

Se tomar�n medidas con el sistema (IMU+Beagleboard) en reposo. Durante un minuto y con una frecuencia de 100 muestras por segundo se tomar�n medidas de aceleraci�n en los tres ejes con el sistema orientado con el eje z alineado con la vertical. Se repite el proceso, para las otras dos direcciones principales.

\subsubsection*{Medidas a Velocidad angular constante}
En primer lugar se debe obtener la velocidad angular del tocadisco. Se toma el tiempo que demora el tocadiscos en dar 10 vueltas completas. Se ubica el sistema a una distancia conocida del centro del tocadiscos. Se calcula la aceleraci�n centr�peta en ese punto. Se toman medidas durante un minuto con una frecuencia de 100 muestras por segundo con las tres orientaciones posibles. 
Se repite el experimento a una distancia distinta. 

\subsubsection*{Medidas de aceleraci�n en un movimiento oscilatorio}
Se mide la masa del sistema.Se determina la constante el�stica del sistema midiendo el estiramiento del resorte al unir el sistema al resorte. Se calcula la aceleraci�n m�xima del sistema si el sistema parte del reposo y estirado 20cm medidos desde la nueva posici�n de equilibrio. Se consideran los m�ximos y m�nimos de las aceleraciones obtendias durante 30 segundos a una tasa de 100 muestras por segundo en los tres ejes.

\subsubsection*{Modelos}
A partir de las medidas obtenidas y los valores te�ricos de las aceleraciones en cada experimento realizado se propone una aproximaci�n lineal y una c�bica. Se obtienen los par�metros de dichos modelos con el m�todo de m�nimos cuadrados.

\subsection{Parte II: Definici�n del modelo}
Se repite el experimento de medir la aceleraci�n de una oscilaci�n. Se ajusta la curva seg�n los modelos propuestos. Se opta por el modelo que de una mejor respuesta sabiendo que la misma debe ser sinusoidal.

\subsection{Parte III: Determinaci�n de no idealidades}
A esta altura se intenta caracterizar las dos fuentes principales de no idealidades de los aceler�metros: El ruido y el Bias.

\subsubsection*{Ruido}
Se modela el ruido propio del aceler�metro como gaussiano de media nula. Se toman medidas en reposo en los tres ejes durante 10 minutos. A partir de dichas medidas se estima la potencia del ruido. Se dise�a un filtro para disminuir dicho ruido. Se grafican las aceleraciones de todos los experimentos realizados hasta el momento con el filtrado definido en este experimento. Se comparan las nuevas respuestas de los aceler�metros con las obtenidas antes del filtrado. 

\subsubsection*{Bias}
Se toman muestras de 10 minutos en reposo para los tres ejes. Se integran dichas muestras y se observa el error de velocidad y posici�n obtenidos. Se repite el experimento 2 veces m�s con algunas horas de diferencia. Se comparan los resultados obtenidos, se estudia si hay alg�n patr�n. 






\end{document}