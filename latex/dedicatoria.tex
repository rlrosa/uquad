\documentclass[main]{subfiles}
\begin{document}

\chapter*{}
\pagenumbering{roman} % para comenzar la numeracion de paginas en numeros romanos
\begin{flushright}
\textit{
Queremos agradecer en primer lugar a nuestras familias, amigos, novias 
y al novio de Rodrigo por el apoyo que nos han brindado a lo largo de
toda la carrera y fundamentalmente durante este proyecto.}
\end{flushright}

\begin{flushright}
\textit{
Dentro del instituto queremos agradecer a Leonardo Steinfeld por prestarnos un microcontrolador MSP430F5438 y una placa de 
conversi\'on de niveles l\'ogicos, a Pablo Toscano por prestarnos un tac\'ometro y a Sergio y Roberto del Taller que nos ayudaron constantemente con los distintos dispositivos que tuvimos que implementar.
}
\end{flushright} 

\begin{flushright}
\textit{
Por \'ultimo queremos agradecer al panadero por prestarnos en varias ocasiones la balanza para realizar la caracterizaci\'on de los motores.
}
\end{flushright}



\chapter*{Res\'umen}
\begin{flushright}
\textit{Presentamos en este trabajo la implementaci\'on de un UAV con arquitectura de cuadric\'optero. Se presentan en esta documentaci\'on la caracterizaci\'on y el modelado de un cuadric\'optero comercial, la integraci\'on del mismo con una inteligencia adquirida y sensores. Se presentan adem\'as los detalles del filtro de Kalman implementado para la estimaci\'on del estado y los detalles del controlador LQR con acci\'on integral -implementados. Dichos algoritmos fueron testeados exhaustivamente con diversas simulaciones y pruebas sobre el sistema real, las cuales tambi\'en se presentan. Se desarroll\'o tambi\'en un simulador en \emph{MatLab} para probar los algoritmos de control y se escrib\'o un manual de usuario del mismo y otro sobre el uso de la plataforma desarrollada.
}
\end{flushright}

El proyecto tiene por objetivo fundamental el diseño y la integración de un sistema de control que permita el vuelo autónonomo de un cuadricóptero comercial radiocontrolado. En particular se realizó la caracterizaci\'on y el modelado del cuadric\'optero comercial TurboAce X720,  se le agregó inteligencia adicional y sensores para obtener una estimación de las variables de estado que describen al sistema utilizando entre otros un filtro de Kalman extendido. La inteligencia se utiliza además para definir y ejecutar las acciones de control necesarias, el controlador implementado es un controlador LQR con efecto integral.   
\end{document}
