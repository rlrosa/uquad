\documentclass[main]{subfiles}
\begin{document}

\appendix

\cleardoublepage
\addappheadtotoc
\appendixpage
\renewcommand{\appendixname}{Anexo}
\renewcommand{\appendixtocname}{Anexo}
\renewcommand{\appendixpagename}{Anexos}
\chapter{C�lculo de los tensores}


Bajo estas suposiciones las cantidades de inter�s en lo que refiere a las dimensiones de sistema son:

\begin{itemize}
\item Radio de la esfera central (R)
\item Largo de las varillas (L)
\item Radio de los motores(r)
\end{itemize}


\subsection{Masa del sistema}
La masa de los objetos que componen al sistema son:

\begin{table}[H]
\centering
\begin{tabular}{p{80pt}|p{80pt}|p{80pt}|p{80pt}|} 
\cline{2-3}
& \cellcolor[gray]{0.8} \textbf{Masa por elemento} 
& \cellcolor[gray]{0.8} \textbf{Cantidad} 
& \cellcolor[gray]{0.8} \textbf{Masa total} \\ \cline{2-4} \hline
\multicolumn{1}{|p{80pt}|}{\cellcolor[gray]{0.8}\textbf{Esfera Central}} 
& $M_E$  & 1 & $M_E$ \\ \hline
\multicolumn{1}{|p{80pt}|}{\cellcolor[gray]{0.8}\textbf{Varilla}} 
& $M_V$ & 4 &$4 M_V$ \\ \hline
\multicolumn{1}{|p{80pt}|}{\cellcolor[gray]{0.8}\textbf{Motores}} 
&$M_M$  & 4 & $4 M_M$\\ \hline
\multicolumn{1}{|p{80pt}|}{\cellcolor[gray]{0.8}\textbf{Masa Total}} & M\\
\cline{1-2}
\end{tabular}
\caption{Masas de los objetos que componen al sistema}
\label{tab:masas}
\end{table}



\subsection{Tensor de inercia del sistema}

\label{tensores}
