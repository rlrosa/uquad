\documentclass[main]{subfiles}

\begin{document}

\chapter{Recursos}
\label{chap:anexo_costos}
En el presente anexo se detallan los recursos utilizados durante el proyecto. Se realiza un an\'alisis desde el punto de vista del costo econ\'omico y de las horas hombre dedicadas.

\section{Costos}

\begin{table}[H]
\begin{center}
\rowcolors{1}{gray!20}{}
\begin{tabular}{|p{5cm}|p{3cm}|}
\hline
Plataforma F\'isica $^{(*)}$  & 819.08 USD  \\
\hline
Beagleboard & 149.00 USD \\
\hline
Beaglejuice & 88.73 USD\\
\hline 
Mongoose 9DOF & 115.00 USD\\
\hline
GPS & 34.99 USD\\
\hline
WiFi & 44.87 USD\\
\hline
Otros $^{(**)}$ & 100.00 USD\\
\hline
Total & 1351.67 USD \\
\hline
\end{tabular}
\label{tab:acc-anexo}
\end{center}
\begin{center}
$^{(*)}$Incluye transmisor, receptor y una bater\'ia\\
$^{(**)}$ Acr\'ilico, cobre, componentes electr\'onicos, aluminio, etc.
\end{center}
\caption{Costos de las partes que componen el sistema}
\end{table}

\subsection{Costos Ocultos}
\label{sec:anexo_costos-costos-ocultos}

Durante el proyecto se utilizaron diversos dispositivos que no fue necesario adquirar, ya que se encontraban disponibles en el laboratorio:

\begin{itemize}
\item \textbf{Osciloscopio:} - 500 USD\footnote{Precio promedio. Los precios oscilan entre 100 USD (USB, requiere de una PC) hasta 3500 USD. Las necesidades del proyecto se cubren con un modelo sencillo.} Es fundamental para trabajar con circuitos electrónicos.
\item \textbf{Soldador con microscopio:} Se utilizó un soldador precisión equipado con un microscopio para trabajar con componentes SMD.
\item \textbf{Fuente DC 5V:} Utilizada para alimentar la electrónica.
\item \textbf{Tocadiscos:} Utilizados durante la calibración del giróscopo.
\item \textbf{Equipo de Aire Acondicionado:} Para las compensaciones por temperatura.
\item \textbf{Software:} Computadoras con MatLab.
\end{itemize}

Algunas inversiones que, a priori, no serían necesarias si se quisiera repetir partes del proyectos:

\begin{itemize}
\item \textbf{Analizador lógico} - $90 USD$ - Si hay que decodificar otro protocolo $I^2C$, es fundamental contar con un analizador lógico. Durante el proyecto se invirtió mucho tiempo en intentar implementar uno, y al final se optó por comprar uno hecho.
\item \textbf{Protecciones en aluminio} - \$1500 - Contando con ESCs confiables, es probable que estas protecciones no sean necesarias, y que sea conveniente sacarlas para aumentar la carga que queda disponible y para disminuir el rozamiento con el aire.
\end{itemize}

Gastos adicionales:

\begin{itemize}
\item \textbf{Envíos:}
  \begin{itemize}
  \item Hubo que enviar una Beagleboard y una Beaglejuice a USA para hacer valer la garantía, el costo de los envíos fueron de 100 USD y 50 USD respectivamente.
  \item Envíos desde China tienen un costo de aproximadamente 30 USD.
  \end{itemize}
\item \textbf{Aduana:} Los gastos de aduana fueron de aproximadamente 200 USD. Algunos debido a que por problemas de tiempo no se hicieron los trámites necesarios para evitar gastos (por tratarse de materiales para la educación).
\end{itemize}

\subsection{Elección de compras - Análisis de alternativas}
\label{sec:anexo_costos-analisis-de-alternativas}

Antes de efectuar las compras, se analizaron las diversas opciones disponibles en el mercado. A continuación se resumen las conclusiones de dicho análisis.

\subsubsection{Instrumentación}
\label{sec:costos-instrumentacion}

La idea original era adquirir una IMU equipada solamente con acelerómetros y giróscopos, y un GPS. Trabajando con esta configuración se descubrió que el GPS es muy poco preciso, y no es posible determinar la orientación (Yaw) ni la altura del sistema con un error razonable (del orden de 5$^o$ y 50cm respectivamente). Se optó por incorporar más sensores, lo que llevó a adquirir la Mongoose, que se utilizó en el diseño final. Incorpora un magnetómetro y un barómetro.

En el mercado (al dia de hoy) no hay ninguna IMU que compita con la Mongoose. Todas traen sensores con especificaciones muy similares, la Mongoose cuesta menos que la competencia\footnote{Ejemplo: Razor 9DoF de Sparkfun - \url{https://www.sparkfun.com/products/10736?}} y trae más sensores, trabaja a la misma frecuencia de muestreo, se comunica mediante el mismo tipo de interfaz, etc.

Cabe destacar que de tener que repetir esta etapa, se buscaría una IMU capaz de funcionar como sistema de estabilización, efectuando las acciones de control básicas. En el sistema existente esto no es posible, debido a que la direccion $I^2C$ del acelerómetro coincide con la de uno de los ESCs, lo que impide que la Mongoose maneje los motores. Esto se podría resolver reemplazando los ESCs por otros cuyo código fuente estuviese disponible y permitiera cambiarles las dirección $I^2C$.

\subsubsection{Inteligencia}
\label{sec:anexo_costos-inteligencia}

Como inteligencia se buscó una placa de desarrollo, ya que no era de interés construir una a partir de un micro. Se buscó que la communicación fuese simple, al igual que la programación. Las dos opciones que se manejaron fueron la Beagleboard y la \textit{Gumstix Overo Fire}\footnote{\url{https://www.gumstix.com/store/product_info.php?products_id=227}}, ambas corren un sistema operativo Linux. Se optó por la Beagleboard por precio y disponibilidad de puertos de communicación. La Gumstix costaba casi el doble y requería de una placa adicional para habilitar la communicación mediante USB.

Hoy en dia en el mercado existen opciones de inteligencia que por $25USD$ brindan la misma capacidad de procesamiento que la Beagleboard\footnote{Ver \url{http://www.raspberrypi.org}.}.

\subsubsection{Plataforma}
\label{sec:anexo_costos-plataforma}

Los criterios que usaron para la elección de la plataforma fueron el costo, la capacidad de carga, y el tiempo de vuelo. Se encontraron tres candidatos.

\begin{table}[H]
\begin{tabular}{p{40pt}|p{70pt}|p{160pt}|p{120pt}|} 
\cline{2-4}
& \cellcolor[gray]{0.8} \textbf{GAUI 330X} 
& \cellcolor[gray]{0.8} \textbf{XAircraft X650} 
& \cellcolor[gray]{0.8} \textbf{Turbo Ace X720} \\ \cline{2-4}
\hline
\multicolumn{1}{|p{40pt}|}{\cellcolor[gray]{0.8}\textbf{Peso}} 
& $700g$ & Versi\'on de fibra de vidrio: $1100g$. Versi\'on de fibra de carbono: $950g$. & $990g$ \\
\hline
\multicolumn{1}{|p{40pt}|}{\cellcolor[gray]{0.8}\textbf{Carga \'util}} 
& $500g$ & Versi\'on de fibra de vidrio: $700g$. Versi\'on de fibra de carbono: $850g$. & $1300 g$ \\ 
\hline 
\multicolumn{1}{|p{40pt}|}{\cellcolor[gray]{0.8}\textbf{Tiempo de vuelo}} 
& Con bater\'ia de $2200 mAh$ vuela entre 7 y 20 minutos & Vuela 12 minutos con bater\'ia de $2200mAh$ y carga menor a $1.5kg$ & Con bater\'ia de $2200mAh$ vuela 15 minutos a carga nominal y puede llegar a la media hora de vuelo con una bater\'ia de $10.000mAh$ \\
\hline
\end{tabular}
\caption{Comparaci\'on peso-carga \'util y tiempo de vuelo sin carga.}
\label{tab:peso}
\end{table}

Se optó por el Turbo Ace X720, fundamentalmente por la capacidad de carga.

Hoy en dia es posible adquirir ``esqueletos'' de cuadric\'optero a precios inferiores a los $20USD$\footnote{Ver \url{www.himodel.com}.}. Si se elimina la posibilidad del vuelo a control remoto, se pueden adquirir ESCs, motores y h\'elices por $300 USD$.

Si se desea hacer trabajo de más alto nivel, por ejemplo navegación, localización, etc, y no es de interés trabajar en el sistema de control se puede optar por el Parrot ARDrone\footnote{\url{https://projects.ardrone.org/}}, es una plataforma robusta que se programa utilizando una API abierta, y cuesta alrededor de $300 USD$.

\section{Horas hombre}
Previo al comienzo del proyecto se realiz\'o una divisi\'on de tareas y una estimaci\'on de la duraci\'on de cada una de ellas. Dicha estimaci\'on fue reformulada al cabo de algunos meses de comenzado el proyecto y se agregaron y reformularon tareas. En la tabla \ref{tab:wbs} se detallan las actividades planificadas, el tiempo estimado y el tiempo que insumieron efectivamente.\\

Se explicara fundamentalmente las razones por las cuales algunas de las tareas insumieron un tiempo muy diferente al estimado inicialmente. En la actividad elecci\'on de hardware se incluye adem\'as la compra de los elementos, en este caso se dilat\'o un poco fundamentalmente debido a la necesidad de coordinar con la contaduria de la Facultad de Ingenier\'ia un horario para realizar las compras.\\

La caracterizaci\'on de los motores fue estimada en mala forma. En primer lugar se debe alcarar que por mal manejo de los componentes se quem\'o un ESC, paralizando por un tiempo dicha actividad. Una vez que se obtuvo un remplazo, el an\'alisis del protocolo para comandar los motores insumi\'o una gran cantidad de tiempo. La raz\'on principal fue la necesidad de un ``sniffer'' para realizar el an\'alisis del protocolo $I^2C$. Al no disponerse de uno se buscaron soluciones alternativas durante varias semanas. Luego de un largo esfuerzo se opt\'o por adquirir uno.\\

La calibraci\'on de la instrumentaci\'on tomo m\'as tiempo de lo deseado. La principal dificultad fue la calibraci\'on del gir\'oscopo, se idearon formas de calibrar el gir\'oscopo que parecian adecuadas, pero su realizaci\'on tom\'o demasiado tiempo. Adem\'as no se obtuvieron resultados acordes.\\

En lo que respecta al generador de rutas, la raz\'on por la cual el tiempo dedicado es muy inferior al estimado es que inicialmente se hab\'ia pensado en un generador eficiente en alg\'un sentido. El generador implementado es muy b\'asico y no es \'optimo en ning\'un sentido.\\

El mayor error en cuanto a la planificaci\'on fue el tiempo dedicado al testeo. Se encontraron diversas dificultades durante esta etapa. En primera instancia exist\'ia un problema con los tiempos de acci\'on del sistema, este problema se detect\'o luego de analizar muchas posibilidades. Una vez corregido el sistema comenz\'o a funcionar en forma m\'as adecuada en las purebas de un s\'olo grado de libertad. A la hora de las pruebas de vuelo el ``glich'' de los ESCs caus\'o la rotura de la plataforma en m\'as de una ocasi\'on. En los 120 d\'ias se incluye el tiempo que tomaron las reparaciones y los ajustes de las matrices de realimentaci\'on y de Kalman, ya que existen diferencias entre las plataformas de las que se dispon\'ia y dichos par\'ametros no funcionaban de igual forma en los dos sistemas.\\ 

\begin{table}
\begin{tabular}{|p{150pt}|c|c|}
\hline
\cellcolor[gray]{0.8} Actividad & \cellcolor[gray]{0.8} Tiempo estimado & \cellcolor[gray]{0.8} Tiempo insumido\\
\hline
Definici\'on del hardware a adquirir & 27 d\'ias & 40 d\'ias\\
\hline
Caracterizaci\'on de los motores & 9 d\'ias & 30 d\'ias\\
\hline
Modulo de comando de los motores & 9 d\'ias &  \\
\hline
Programaci\'on del microprocesador para comunicarse con la instrumentaci\'on & 20 d\'ias & \\
\hline
Calibraci\'on de la instrumentaci\'on &17 d\'ias & 30 d\'ias \\
\hline
Comunicaci\'on WiFi &20 d\'ias & \\
\hline
Armado del dispositivo & 9 d\'ias & \\
\hline
Modelo f\'isico & 16 d\'ias & 20 d\'ias \\
\hline
Estudio de vuelo (linealizaci\'on) & 11 d\'ias & 7 d\'ias \\
\hline
Generador de rutas & 30 d\'ias& 3 d\'ias \\
\hline
Diseño del simulador &30 d\'ias & 30 d\'ias \\
\hline
Diseño de los algorimos de control & 36 d\'ias & 40 d\'ias \\
\hline
Simulaci\'on, testeo y debugging de algor\'itmos & 7 d\'ias & 15 d\'ias \\
\hline
Programaci\'on del microprocesador & 32 d\'ias & \\ 
\hline
Pruebas sobre el sistema real & 15 d\'ias & 120 d\'ias \\
\hline
\end{tabular}

\caption{Actividades planificadas}
\label{tab:wbs}
\end{table} 


\end{document}