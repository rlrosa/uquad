\documentclass[main]{subfiles}

\begin{document}

\chapter{Código}
\label{chap:codigo}

El objetivo de esta sección es explicar como utilizar el código con el que vuela el cuadricóptero.

\section{Esquema general}
\label{sec:codigo:esquema-general}

% fotito
% nombrar bloques
% ref a que lee entre lineas en C y jodete
% medio de communicacion

\section{Ejecución}
\label{sec:codigo:ejecucion}

%TODO
% como correr

\subsection{Comunicación}
\label{sec:codigo:comunicacion}

%TODO
% wifi + ssh
% piques de clave pub
% adhoc
% dhcp

\subsection{Tiempos}
\label{sec:codigo:tiempos}

%TODO
% logging
% IO
% cualquier cosa te caga la vida

\subsection{Consideraciones de seguridad}
\label{sec:codigo:consideraciones-de-seguridad}

% check_net
% falla el wifi cuando le pinta
% no saturar el ssh/wifi

\section{Salida - Logs}
\label{salidas-logs}

% quien es quien
% limite del logger

\section{Bloques}
\label{sec:codigo:bloques}

\section{Compilación}
\label{sec:codigo:compilacion}

%TODO
% cross
% nativo

\subsection{Git}
\label{sec:codigo:git}

%TODO
% server en github, o desde una pc cualquiera
% espacio en la SD

\section{Debugging}
\label{sec:codigo:debugging}

%TODO

\subsection{Ejecutar a partir de un log}
\label{sec:codigo:ejecutar-a-partir-de-un-log}

%TODO

\subsubsection{C Vs. MatLab}
\label{sec:codigo:c-vs-matlab}

%TODO

\section{Kernel}
\label{sec:codigo:kernel}

%TODO
% explicar xq tuvimos que armar uno

\subsection{\textit{Cross-compiling - Bitbake+OpenEmbedded}}
\label{sec:codigo:cross-compiling-bitbake-oe}

%TODO
%  piques sobre como no joderse tanto manejar el bitbake+oe

\end{document}