\documentclass[main]{subfiles}
\begin{document}

\chapter{Testeo del Beagleboard y el \textit{IMU}}

Durante la elecci�n del hardware se busc� informaci�n sobre el \textit{Beagleboard} (de ahora en m�s ``\textit{\textit{BB}}'') y sobre el \textit{Atomic IMU} (de ahora en m�s ``\textit{IMU}''), y se concluy� que ser�an suficientes para los requerimientos del proyecto. En este cap�tulo se explican las pruebas realizadas sobre ambas placas. Los objetivos de dichas pruebas fueron:

\begin{itemize}
\item Verificar el correcto funcionamiento de las placas.
\item Confirmar que cumplen con los requerimientos del proyecto.
\end{itemize}

\section{Testeo del Beagleboard}
\label{sec:testeo_del_beagleboard}

El microcontrolador a bordo del \textit{BB} (de ahora en m�s ``\textit{omap}'') es suficiente para manejar el loop de control, y tiene un \textit{DSP}, por lo que la capacidad de procesamiento no deber�a ser un problema. La evaluaci�n de dicha capacidad se har� m�s adelante, cuando se cuente con software adecuado. Por ahora solamente se analiz� la disponibilidad de se�ales \textit{PWM}.

\subsection{\textit{PWM}}
\label{sec:beagleboard_pwm}

El \textit{omap} cuenta con 4 m�dulo para \textit{PWM} implementados en hardware. Para poder utilizarlos es necesario deshabilitar una funci�n de ahorro de energ�a en el kernel. Se utiliz� \href{http://www.openembedded.org}{OpenEmbedded} para compilar un kernel adecuado.

Se utiliz� un driver hecho por \href{https://github.com/scottellis/omap3-pwm}{Scottellis} para manejar el acceso a registros en el \textit{omap}. Puede que no sea necesario manejar los \textit{PWM} desde el kernel, ya que la implementaci�n est� en hardware, por lo que se est� considerando pasa a usar un programa en \textit{user space}, evitando los riesgos de trabajar a bajo nivel. Un error a nivel de kernel podr�a dejar deshabilitar el \textit{BB}, lo cual llevar�a inevitablemente a perder el control del cuadric�ptero.

Solamente 3 de los 4 \textit{PWM} est�n mapeados a pines accesibles en el \textit{BB}. Para tener acceso al cuarto \textit{PWM} es necesario modificar la configuraci�n de muxing en el \textit{omap}. Habilitar el cuarto \textit{PWM} trae al menos dos problemas:
\begin{enumerate}
\item Queda inutilizable el integrado \textit{U14}, que es un \textit{Hi-Speed USB 2.0 Transceiver}.
\item Hace falta conectar un cable a un punto en la mitad del \textit{BB} para ver la se�al de \textit{PWM}.
\end{enumerate}

Se encontraron comentarios en foros sobre posibles problemas al utilizar los \textit{PWM} en conjunto con el \textit{DSP}, pero no se encontr� informaci�n que confirmara esto en las hojas de datos, y todav�a no se cuenta con software adecuado para hacer pruebas.

Se determin� que es posible utilizar el \textit{BB} para generar 4 se�ales \textit{PWM}, pero con un costo que puede ser importante. Todav�a se est� analizando si es posible mapear el cuarto \textit{PWM} a otro pin del \textit{omap}. Esto

Alternativas bajo estudio:

\begin{itemize}
\item Mapear el cuarto \textit{PWM} a otro pin del \textit{omap}:
  \begin{itemize}
  \item Esto permitir�a utilizar el \textit{U14}.
  \item Seguramente se perder�a alguna otra cosa.
  \end{itemize}
\item Utilizar alguno de los 6 \textit{PWM} disponibles en el microcontrolador a bordo del \textit{IMU}:
  \begin{itemize}
  \item Permitir�a utilizar todas las funcionalidades del \textit{BB}.
  \item El \textit{IMU} puede quedar sobrecargado, resultando en p�rdidas de datos de los sensores.
  \item El tiempo que tarda en llegar una orden emitida por el loop de control a su destino puede ser un problema Dicha orden tiene que seguir el siguiente camino:
    \begin{enumerate}
    \item Sale del \textit{BB} y va al \textit{IMU}.
    \item El \textit{IMU} configura el \textit{PWM} adecuadamente.
    \item La se�al de \textit{PWM} va al \textit{ESC} correspondiente.
    \item El \textit{ESC} actua sobre el motor correspondiente.
    \end{enumerate}
  \end{itemize}
\item Utilizar una tercer placa para generar los \textit{PWM}.
\end{itemize}

\newpage

\section{Testeo del \textit{IMU}}
\label{sec:testeo_imu}

%TODO 

\end{document}